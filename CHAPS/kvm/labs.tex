\clearpage\section{Labs}\begin{Lab}

\begin{exe} {Verify system is ready for KVM }

   Verify the CPU chip extensions are available and enabled to support virtualization. 
   Confirm the required packages are available. 

   This exercise requires a 64 bit architecture system with a minimum of 4GiB 
	if memory and 5GiB of
	free disk space. Internet connectivity is recommended but not critical. 

   It is common practice to run virtual machines as a non-root user, but in this
   exercise we will execute the commands as root.

   \begin{sol}
      \begin{enumerate}
         \item	Verify the cpu has virtualization support enabled.
		 \begin{raw}
# grep -e svm -e vmx /proc/cpuinfo 
		\end{raw} 
		      The output should have one of the flags highlighted, either 
		      \textbf{svm} or \textbf{vmx}. If no output was produced, verify 
		      in your \textbf{BIOS} that virtualization is supported. 
		  
	\item Get the \textbf{knk-resources.tar} file. The presenter will 
		have the information as to how to obtain the file. 
	  \item Install required packages:		      
         \begin{itemize}
            \item
            On \textbf{CentOS7}:
            \begin{raw}
# yum install qemu-kvm libvirt virt-manager virt-install virt-viewer 
	    \end{raw}
            \item
            On \textbf{OpenSUSE}:
            \begin{raw}
# zypper install qem-kvm libvirt virt-manager virt-viewer bridge-utils 
            \end{raw}
            \item
            On \textbf{Ubuntu}:
            \begin{raw}
# apt-get install qemu-kvm libvirt-bin virtinst \
		    virt-manager libosinfo-bin virt-viewer 
            \end{raw}
         \end{itemize}

         \item
         Verify and optionally start and enable the \textbf{libvirtd} service.

         \begin{raw}
# systemctl status libvirtd
# systemctl start libvirtd
# systemctl enable libvirtd
# systemctl status libvirtd
         \end{raw}

 	\item 
	If using \textbf{Ubuntu} or \textbf{Centos} please log in 
		      with \textbf{GNOME Xorg} not \textbf{GNOME Wayland}.
	\item 
	If using \textbf{openSUSE} you may have to add 
		      \textbf{export LIBVIRT\_DEFAULT\_URI=qemu:///system} \\
		      to 
		      \filelink{/etc/bash.bashrc.local}.  
		      
		      This will allow a regular user 
		      to administer all the virtual machines. 
		      \begin{raw}
# echo "export LIBVIRT_DEFAULT_URI=qemu:///system"  >> /etc/bash.bashrc.local
			\end{raw}
	\item 
	If challenged for the administrator password using \textbf{virsh} commands 
		      similar to the following screenshot.


\begin{figure}[H]
      \includegraphics[height=3.2in]{IMAGES/libvirt-auth.png}
      \caption{libvirt authentication prompt}
   \end{figure}



	Set the \textbf{unix\_sock\_group} variable in \filelink{/etc/libvirt/libvirtd.conf}
		      to \textbf{libvirt}, it may be added to the bottom of the file like:
		      \begin{raw}
# echo 'unix_sock_group = "libvirt" ' >> /etc/libvirt/libvirtd.conf
			\end{raw}
			Then add the additional supplemental group \textbf{libvirt} group to your user:
			\begin{raw}
# usermod -a -G libvirt <username> 
			\end{raw} 

      \end{enumerate}

   \end{sol}

\end{exe}

\begin{exe} {Create and run a Virtual Machine based on an image}

	Often a virtual disk is used for distributing a 
	pre-configured environment, for this exercise a pre-installed
	operating system image file is available in the
	\textbf{knk-resources.tar} file.
	The image file is called \textbf{tiny2.vmdk}.


	This exercise will create a virtual machine 
	using an \textbf{xml} file as input to \textbf{virsh}.

	The parameters for the virtual machine are:
	\begin{itemize}
		\item VM name is \textbf{tiny2}
		\item 1 virtual CPU
		\item 512M memory
		\item the virtual disk is \file{/var/lib/libvirt/images/tiny2.vmdk}
		\item use \textbf{virt-viewer} to access the virtual machine
	\end{itemize}

	In this exercise a pre-installed disk \file{tiny2.vmdk} contains a 
	operating system and an application to be made available. 
	A \textbf{virsh define} command is 
	to be used with a \textbf{xml} configuration file. 

		
	\begin{sol}
	
		\begin{enumerate}
		\item Create a directory to store the exercise 
			resources in and extract the tarball.
			\begin{raw}
# mkdir  /var/tmp/knk 
# cd /var/tmp/knk
# tar -xvf <download directory>/knk-resources.tar
			\end{raw}

		\item Create a copy of the disk image 
			and store it at 
			\file{/var/lib/libvirt/images/tiny2.vmdk}
			\begin{raw} 
# cp /var/tmp/knk/tiny2.vmdk  /var/lib/libvirt/images/ 
			\end{raw}

		\item

        \textbf{Note:} As an added extra credit adventure,
        some distro's do not support writing to a \textbf{vmdk}
        image file. If \textbf{CentOS7} is being used for the host,
        the tiny2.vmdk must be converted to qcow2. The
        conversion is done with the \textbf{qemu-img} command.

        This will convert the file:
                \begin{raw}
# cd /var/lib/libvirt/images/
# qemu-img convert -O qcow2 tiny2.vmdk tiny2.qcow2
                \end{raw}
        You will have to remember to adjust the file
        name in the rest of the lab exercise.


		\item Create an \textbf{xml} configuration file. 
			(sample files are included in the
				\filelink{/var/tmp/knk/} directory)

		\textbf{Note:} There are three xml sample files 
				in the resources directory, the names 
				relate to the distro the host computer
				is running:
				\begin{itemize}
				\item tiny2.xml-ubuntu  
				\item tiny2.xml-centos
				\item tiny2.xml-suse
				\end{itemize}
				Copy or rename the appopiate 
				\textbf{xml} file to be \textbf{tiny2.xml} 
				\begin{raw}
# cp tiny2.xml-ubuntu tiny2.xml
				\end{raw}
				The name and extension is not manditory
				but consistant naming helps with the exercise.


				Please feel free to compare the xml files as 
				they illustrate minor changes in 
				the distributions.

				An example of the configuratioin file is: 
			\rawfile{tiny2.html.inc}

		\item define the virtual machine 
			\begin{raw}
# virsh define tiny2.xml
			\end{raw}
		\item verify the configuration file 
			\begin{raw}
# virsh dumpxml tiny2 
			\end{raw}
		\item start the virtual machine 
			\begin{raw}
# virsh start tiny2
			\end{raw}
		\item connect to the virtual machine
			\begin{raw}
# virt-viewer tiny2 
			\end{raw}
		\item exit the virt-viewer and verify the state of the VM
			\begin{raw}
# virsh list --all 
			\end{raw}
		\item try a graceful shutdown of the VM, 
			success or failure will depend on the OS
			\begin{raw}
# virsh shutdown tiny2 
			\end{raw}
		\item issue a forced shutdown of the VM 
			if the graceful shutdown did not work
			\begin{raw}
# virsh destroy tiny2 
			\end{raw} 
		\end{enumerate}

	\end{sol}
\end{exe}

\begin{exe}  {Create a Virtual Machine using virt-install}

	
	In the previous exercise \textbf{virsh} 
	was used to create the configuration using an xml file as input.


	The object of this exercise is to simplify the install procedure 
	by using the \textbf{virt-install} tool. 

	One feature of \textbf{virt-install} is the 
	automatic configuration of a default network connection. 
	This feature can be overridden if desired. In this exercise the default 
	network connection will be used. 

	 The parameters for the virtual machine are:
        \begin{itemize}
                \item name is \textbf{tiny3}
                \item 1 virtual CPU
                \item 512M memory
                \item the virtual disk is \file{/var/lib/libvirt/images/tiny3.vmdk},
			a copy of the previously used \textbf{tiny2.vmdk}
		\item use a network connection called \textbf{default}
                \item use \textbf{virt-viewer} to access the virtual machine
	\end{itemize} 

	Once the \textbf{VM} has been created and tested that it runs, compare the active 
	configuration that \textbf{libvirt} has stored. The configuration created by
	\textbf{virt-install} will have many more default options enabled, including 
	an active network connection NAT'd to the default adapter on the host computer.

	\begin{sol}
		\begin{enumerate}
		\item  Create a copy of \textbf{tiny2.vmdk} and place it in
			\file{/var/lib/libvirt/images/tiny3.vmdk}
			\begin{raw}
# cp /var/lib/libvirt/images/tiny2.vmdk /var/lib/libvirt/images/tiny3.vmdk
			\end{raw}
		\item Confirm or create a user called \textbf{dnsmasq}, it is used by
			libvirt. 
		\begin{raw}
# grep dnsmasq /etc/passwd
		\end{raw} 
			if the user \textbf{dnsmasq}  does not exist, create it.
		\begin{raw}
# useradd dnsmasq
		\end{raw} 
		\item Verify a virtual network called \textbf{default} 
			exists and is set to autostart.
				\begin{raw}
# virsh net-list --all 
 Name                 State      Autostart     Persistent
----------------------------------------------------------

				\end{raw} 
			If the \textbf{default} network does not exist, use
				\textbf{virsh} to define the network from the xml
				file \filelink{default-net.xml}. 
				
				The contents of the 
				\filelink{default-net.xml} are: 
				\begin{raw}
<network>
  <name>default</name>
  <uuid>96ea6db6-cfb9-48fa-ad6a-bd21ab81c0d3</uuid>
  <forward mode='nat'>
    <nat>
      <port start='1024' end='65535'/>
    </nat>
  </forward>
  <bridge name='virbr0' stp='on' delay='0'/>
  <mac address='52:54:00:95:cb:bd'/>
  <domain name='default'/>
  <ip address='192.168.122.1' netmask='255.255.255.0'>
    <dhcp>
      <range start='192.168.122.128' end='192.168.122.254'/>
    </dhcp>
  </ip>
</network>
				\end{raw}

				Based on the default-net.xml file, the command to 
				define the network is:
				\begin{raw}
# virsh net-define default-net.xml 
Network default defined from default-net.xml
				\end{raw}
				Then set the network to autostart:
				\begin{raw}
# virsh net-autostart default 
Network default marked as autostarted
				\end{raw}
				Verify the network looks good. 
				\begin{raw}
# virsh net-list --all 
 Name                 State      Autostart     Persistent
----------------------------------------------------------
 default              inactive   yes           yes

				\end{raw}
				Start the new network. 
				\begin{raw}
# virsh net-start default
				\end{raw}

		\item  Use the virt-install command to \textbf{import} the existing 
			virtual machine's disk into a new configuration.
			\begin{raw} 
# virt-install \
              --name tiny3 \
              --memory 512 \
              --disk /var/lib/libvirt/images/tiny3.vmdk \
	      --os-type linux --os-variant generic  \
              --import
			\end{raw}
		
		\item 
			Examine the active configurations stored by \textbf{libvirt}. 
			\begin{raw}
# virsh dumpxml tiny3 > /tmp/tiny3.xml.dump
# virsh dumpxml tiny2 > /tmp/tiny2.xml.dump
			\end{raw} 

			Use commands like diff and ls to see the difference in the xml.dump files. 

		\end{enumerate}

	\end{sol}

\end{exe} 

\begin{exe} {Create a VM and install from a CD-rom image} 

	This exercise will demonstrate installation of an 
	operating system from a CD-rom image
	using \textbf{virt-install}. 
	The CD-rom image is available in resources tar
	file installed earlier. Since 
	installation of operating systems can take a
	long time and require access to 
	download updates or repository data a 
	small self contained OS was chosen for 
	this exercise. 

	\textbf{Note:} During the initial start up, 
	a default of a text interface has been added.
	This avoids 
	some interesting issues with the mouse 
	during and after installation of Tiny Core Linux.
	This change has bee added specifically for the 
	confrence environment. If a graphics interface is 
	desire, the \textbf{TinyCore-Plus} installation
	media is recomended.  

 The parameters for the virtual machine are:
        \begin{itemize}
                \item name is \textbf{tinycore}
                \item 1 virtual CPU
                \item 512M memory
		\item 100M new virtual disk image
                \item the virtual disk is 
			\file{/var/lib/libvirt/images/tinycore.qcow2}
                \item the virtual cd is 
			\file{/var/tmp/knk/k-n-k-3.iso}
                \item use \textbf{virt-viewer} 
			to access the virtual machine
	\end{itemize}

	\begin{enumerate}
		\item Copy the CD-rom image to \filelink{/var/lib/libvirt/images}
			directory. 
		\begin{raw}
# cp /var/tmp/knk/k-n-k-3.iso /var/lib/libvirt/images/k-n-k-3.iso  
		\end{raw}
	\item Boot from the CD-rom using \textbf{virt-install} using the options specified below:
		\begin{raw}
$ virt-install  --name tinycore --memory 512 --vcpus 1 \
		--cdrom /var/lib/libvirt/images/k-n-k-3.iso  \
		--disk /var/lib/libvirt/images/tinycore.qcow2,size=0.1 \
		--os-type linux --os-variant generic --boot useserial=on 
		\end{raw}
			A \textbf{virt-viewer} window should pop open. 
		\item When the initial Core Plus screen appears, press the \textbf{enter} key.

			After the boot messages finish, the user friendly Tiny Core 
			command prompt will be visible. 
		\item To launch the installer at the prompt type: 
			\begin{raw}
$ sudo tc-install.sh 
			\end{raw}
		\item Please fill in the installation choices as listed below:
			\begin{itemize}
				\item Core Installation 
					\begin{raw}
c for cdrom, enter to continue 
					\end{raw}
				\item Install type 
					\begin{raw}
f for frugal, enter to continue 
					\end{raw}
				\item Select Target for Installation
					\begin{raw}
1 for whole disk 
					\end{raw}
				\item Bootloader 
					\begin{raw}
y for yes for bootloader 
 				\end{raw}
				\item Install Extensions  
					\begin{raw}
TCE/CDE Directory is left blank, press enter 
				\end{raw}
				\item Disk Formatting Options
					\begin{raw}
Select 3 for ext4
					\end{raw}
				\item Boot Options
					\begin{raw}
No additional boot options, press enter 
					\end{raw}
				\item Last chance before destroying a data on sda
					\begin{raw}
yes, start the install 
					\end{raw}
			\end{itemize}

			The installation should complete in a few seconds. 

		\item When the installation is complete type in: 
			\begin{raw}
sudo reboot 
			\end{raw} 
		\end{enumerate}

\end{exe}


\begin{exe}  {Create a Virtual Machine and install with a text console}
	
	\textbf{Note}: This is an optional lab not intended for completion at the
	conference but rather a take home exercise. You will need to download an install DVD.
	This example uses a \textbf{Debian 9.5.0} DVD as its source and is approximately
	3.4G. The images can be found at:

	\url{https://cdimage.debian.org/debian-cd/current/amd64/iso-dvd/}

	\textbf{Note:} This installation \textbf{will} access the Internet 
	to communicate with the \textbf{Debian} repository servers.
	To avoid the outbound connection and speed up the install,
	shutdown the network access 
	on the host system before starting the installation. 

 The parameters for the virtual machine are:
        \begin{itemize}
                \item name is \textbf{test-loc}
                \item 1 virtual CPU
                \item 512 memory
		\item 1.5G disk 
                \item the virtual disk is going to be manually created, the
			suggested location is 
			\file{/var/lib/libvirt/images/deb9.qcow2}
		\item the image used as the DVD can reside anyplace 
			there is space
                \item use \textbf{virt-viewer} to access the virtual machine
		\item the install and finished installation will be text only
	\end{itemize}

	\begin{sol}
	\begin{enumerate}
		\item Create a disk image with a size of \textbf{1.5GiB} and a 
			type of \textbf{qcow2}. 
			\begin{raw}
# qemu-img create -f qcow2 /var/lib/libvirt/images/deb9.qcow2 1.5g
			\end{raw} 
		\item Use \textbf{virt-install} to create the
			\textbf{VM} and start the install sequence. 
			Answer the \textbf{OS} installation questions for a 
			minimal configuration. 
		\begin{raw}

 virt-install \
	--name deb9 \
	--memory 2048 \
	--disk /var/lib/libvirt/images/deb9.qcow2 --vcpus 2 \
	--os-type linux \
	--os-variant debian9 \
	--boot useserial=on  \
	--cdrom /var/ftp/pub/iso/debian-9.5.0-amd64-DVD-1.iso

		\end{raw}
		\item When the installation is completed, a 
			\textbf{virt-viewer} window should open up 
			with the login prompt for the new \textbf{VM}.
			
	\end{enumerate}

	\end{sol}

\end{exe}

\begin{exe} {Challenge exercise using \textbf{virt-manager}, a KVM management GUI}

	In the progression of configuration tools for KVM from \textbf{xml} files
	to \textbf{virt-install} utility and now to feature packed GUI, 
	\textbf{virt-manager}. 
	Everything done in our exercises can be done with \textbf{virt-manager}. If 
	the configuration was done using libvirt (all of ours were) then
	you should be able to manage all of the previously created VM's. 

	The challenge in this exercise is to repeat the exercise steps 
	using \textbf{virt-manager}.

\end{exe}

\end{Lab}

